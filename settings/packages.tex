% ---------------------------------------------------------------------------- %
% Packages
% ---------------------------------------------------------------------------- %

% Eingabekodierung
\usepackage[utf8]{inputenc}

% Sprache (Deutsch, Neue Rechtschreibung)
\usepackage[ngerman]{babel}

% \usepackage{titlesec}

% Biblatex Settings (APA-Deutsch)
\usepackage[style=apa, autolang=other]{biblatex}
\DeclareLanguageMapping{ngerman}{ngerman-apa}

% Dient zur Auszeichnung und erstellung von Hyperlinks
\usepackage{url}

\usepackage{ 
	graphicx, 	% Ermöglicht das Einbinden von Bildern
	amsmath,	% Standard Mathe-Paket
	amsfonts,	% Ergänzt mathematische Symbole
	paralist,	% Erweiterung der bereits bestehenden Listenumgebungen
	listings,	% Code-Formatierung 
	setspace,	% Zeilenabstand
	geometry,	% Seitenformatierung
	fancyhdr,	% Manipulation von Fuss- und Kopfzeilen
	courier,	% Monospace Schrift
	mathptmx,	% Times New Roman für Lauftext und Überschriften
	xcolor		% Erlaubt das Definieren neuer Farben
}

% Bildunterschriften fromatieren
\usepackage[bf]{caption}
\captionsetup{format=plain}

% Layoutanpassungen 
\geometry{
	a4paper,	% Format (überschreibt documentclass)
%	twoside,	% für doppelseitigen Druck verwenden
	footskip=10mm,	% Position Fusszeile (Abstand zum Text)
	headsep=10mm,	% Position Kopfzeile (Abstand zum Text)
	top=25mm,
	right=25mm,
	bottom=20mm,
	left=30mm,
	footnotesep=1cm	% Abstand Text - Fussnote
}

% Mikrotypografie
\usepackage[
	activate={true,nocompatibility},
	final,
	tracking=true,
	kerning=true,
	spacing=true,
	factor=1100,
	stretch=10,
	shrink=10
]{microtype}

\addtokomafont{disposition}{\rmfamily}

% Umlaute, Akzente,...
\usepackage[T1]{fontenc}

% Anführungszeichen
\usepackage[autostyle=true, german=swiss]{csquotes}

% Erweiterter Hyperlink Support
\usepackage[hidelinks]{hyperref}

% Passt Schriftgrösse der Bildbeschreibung an
\captionsetup[figure]{font=footnotesize,labelfont=footnotesize,labelfont=bf}


% ---------------------------------------------------------------------------- %
% Optionale Packages und Einstellungen
% ---------------------------------------------------------------------------- %

% Ermöglicht das Einfügen von Notizen mittels \todo{}
\reversemarginpar % Notizen links anzeigen
\setlength{\marginparwidth}{2.2cm}
\usepackage[backgroundcolor=white,linecolor=black,textsize=small]{todonotes}
\makeatletter
\renewcommand{\todo}[2][]{%
    \@todo[tickmarkheight=0.2cm,caption={#2}, #1]{\begin{spacing}{0.6}#2\end{spacing}}%
} 
\makeatother 

% Fügt im Literaturverzeichnis "Verfügbar unter" und optional "Letzter Zugriff" 
% zu .bib entries mit url und urldate hinzu.
\DeclareFieldFormat{formaturl}{Verfügbar unter #1}
\DeclareFieldFormat{formatdate}{Letzter Zugriff: #1}

\newbibmacro*{url+urldate}{%
	\iffieldundef{urlyear}{%
		\printtext[formaturl]{\printfield{url}}\nopunct%
	}{%
		\printtext[formaturl]{\printfield{url}}\adddot\space%
		\printtext[formatdate]{\printurldate}%
	}%
}

% Ergänzt im Literaturverzeichnis den Publisher mit Ort (Ort, Herausgeber)
\DeclareListFormat{publisher}{\printlist{location}\addcomma\space #1}
