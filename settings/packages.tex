% ---------------------------------------------------------------------------- %
% Packages
% ---------------------------------------------------------------------------- %

% Eingabekodierung
\usepackage[utf8]{inputenc}

% Sprache (Deutsch, Neue Rechtschreibung)
\usepackage[ngerman]{babel}

% \usepackage{titlesec}

% Dient zur Auszeichnung und erstellung von Hyperlinks
\usepackage{url}

\usepackage{ 
	graphicx, 	% ermöglicht das Einbinden von Bildern
	amsmath,		% standard Mathe-Paket
	amsfonts,	% ergänzt mathematische Symbole
	paralist,	% Erweiterung der bereits bestehenden Listenumgebungen
	listings,	% Code-Formatierung 
	setspace,	% Zeilenabstand
	geometry,	% Seitenformatierung
	fancyhdr,	% Manipulation von Fuss- und Kopfzeilen
	courier,		% Monospace Schrift
	mathptmx,	% Times New Roman für Lauftext und Überschriften
	xcolor		% erlaubt das Definieren neuer Farben
}

% Layoutanpassungen 
\geometry{
	a4paper,			% Format (überschreibt documentclass)
%	twoside,			% für doppelseitigen Druck verwenden
	footskip=10mm,	% Position Fusszeile (Abstand zum Text)
	headsep=10mm,	% Position Kopfzeile (Abstand zum Text)
	top=25mm,
	right=25mm,
	bottom=20mm,
	left=30mm
}

% Mikrotypografie
\usepackage[
	activate={true,nocompatibility},
	final,
	tracking=true,
	kerning=true,
	spacing=true,
	factor=1100,
	stretch=10,
	shrink=10
]{microtype}

\addtokomafont{disposition}{\rmfamily}

\usepackage[T1]{fontenc}			% Umlaute, Akzente,...
\setstretch{\zab}				% Setzt den Zeilenabstand
\setlength{\parindent}{\abs} 	% Setzt den Absatzeinzug

% Anführungszeichen
\usepackage[autostyle=true, german=swiss]{csquotes}

% Gepunktete Linien auch bei Subsections im Inhaltsverzeichnis:
% \usepackage{tocstyle}
% \newtocstyle[KOMAlike][leaders]{alldotted}{}
% \usetocstyle{alldotted}

\usepackage[]{hyperref}