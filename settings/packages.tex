% ---------------------------------------------------------------------------- %
% Packages
% ---------------------------------------------------------------------------- %

% Eingabekodierung
\usepackage[utf8]{inputenc}

% Sprache (Deutsch, Neue Rechtschreibung)
\usepackage[ngerman]{babel}

% APA Zitierstil
\usepackage[natbibapa]{apacite}
\bibliographystyle{apacite} 

\usepackage{ 
	graphicx, 			% ermöglicht das Einbinden von Bildern
	amsmath,			% standard Mathe-Paket
	amssymb,			% ergänzt mathematische Symbole
	paralist,			% Erweiterung der bereits bestehenden Listenumgebungen
	listings,				% Code-Formatierung 
	setspace,			% Zeilenabstand
	geometry,			% Seitenformatierung
	fancyhdr,			% Manipulation von Fuss- und Kopfzeilen
	courier,				% Monospace Schrift
	mathptmx,		% Times New Roman für Lauftext und Überschriften
	xcolor				% erlaubt das Definieren neuer Farben
}

% Layoutanpassungen
% @link http://mirror.easyname.at/ctan/macros/latex/contrib/geometry/geometry.pdf
\geometry{
	a4paper,				% Format (überschreibt documentclass)
	twoside,				% für einseitigen Druck oneside verwenden
	footskip=10mm,	% Position Fusszeile (Abstand zum Text)
	headsep=10mm,	% Position Fusszeile (Abstand zum Text)
	top=25mm,
	right=25mm,
	bottom=20mm,
	left=30mm
}

% Mikrotypografie
\usepackage[
	activate={true,nocompatibility},
	final,
	tracking=true,
	kerning=true,
	spacing=true,
	factor=1100,
	stretch=10,
	shrink=10
]{microtype}

\addtokomafont{disposition}{\rmfamily}

\usepackage[T1] {fontenc}		% Umlaute, Akzente,...
\setstretch{\zab}						% Setzt den Zeilenabstand
\setlength{\parindent}{7mm} 	% Setzt den Absatzeinzug

% Anführungszeichen
% @link https://de.wikibooks.org/wiki/LaTeX-W%C3%B6rterbuch:_Anf%C3%BChrungszeichen
\usepackage[autostyle=true, german=swiss]{csquotes}
