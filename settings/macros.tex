% ---------------------------------------------------------------------------- %
% Makros
% ---------------------------------------------------------------------------- %

% TITELSEITE

% Titel und Untertitel
\newcommand{\haupttitel}{Raus aus der Höhle}
\newcommand{\untertitel}{Vom Glanz des Sonnenlichts}

% Verfasser/Verfasserin
\newcommand{\verfassergen}{} % "in" für weibliche Form (Verfasserin)
\newcommand{\verfasser}{Platon}

% Adresse
\newcommand{\strasse}{Akademeia 42}
\newcommand{\ort}{Athen}
\newcommand{\tel}{-}
\newcommand{\mail}{-}

% Referent/Referentin
\newcommand{\referentgen}{} % "in" für weibliche Form (Referentin)
\newcommand{\referent}{Prof. Dr. Sokrates}

% Korreferent/Korreferentin
\newcommand{\korreferentgen}{} % "in" für weibliche Form (Korreferentin)
\newcommand{\korreferent}{Xenophon}

% Modul
\newcommand{\modul}{Sokratische Methode}

% Fusszeile (Ort, Jahr)
\newcommand{\titlefooter}{Athen, 403 v. Chr.}

% Literaturverzeichnis umbenennen
\renewcommand\refname{}

% Eigene Makros können nach folgendem Schema erstellt werden:
\newcommand{\yourmacro}{Dein Text}


% ---------------------------------------------------------------------------- %
% Abkürzungen
% ---------------------------------------------------------------------------- %

\newcommand{\zab}{1.3}	% Zeilenabstand
