% ---------------------------------------------------------------------------- %
% MAKROS
% ---------------------------------------------------------------------------- %

% Eigene Makros können nach folgendem Schema erstellt werden:
\newcommand{\yourmacro}{Dein Text}


% ---------------------------------------------------------------------------- %
% Titelseite
% ---------------------------------------------------------------------------- %

% Nicht verwendete Angaben müssen in setup.tex (45-62) auskommentiert werden.

% Titel und Untertitel
\newcommand{\haupttitel}{Raus aus der Höhle}
\newcommand{\untertitel}{Vom Glanz des Sonnenlichts}

% Verfasser/Verfasserin
\newcommand{\verfassergen}{Verfasser:}
\newcommand{\verfasser}{Platon}

% Adresse
\newcommand{\strasse}{Akademeia 42}
\newcommand{\ort}{Athen}

\newcommand{\telgen}{Tel.: }
\newcommand{\tel}{-}

\newcommand{\mailgen}{e-Mail: }
\newcommand{\mail}{-}

% Referent/Referentin
\newcommand{\referentgen}{Referent:}
\newcommand{\referent}{Prof. Dr. Sokrates}

% Korreferent/Korreferentin
\newcommand{\korreferentgen}{Korreferent:}
\newcommand{\korreferent}{Xenophon}

% Modul
\newcommand{\modulgen}{Modul:}
\newcommand{\modul}{Sokratische Methode}

% Fusszeile (Ort, Jahr)
\newcommand{\titlefooter}{Athen, 403 v. Chr.}


% ---------------------------------------------------------------------------- %
% Diverse
% ---------------------------------------------------------------------------- %

% Literaturverzeichnis umbenennen
\renewcommand\refname{}

% Zeilenabstand
\newcommand{\zab}{1.3}
